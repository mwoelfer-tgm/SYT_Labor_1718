%!TEX root=../document.tex

\section{Einführung}
Als Lastverteilung (englisch Load Balancing) bezeichnet man in der Informatik der Verteilung von umfangreiche Berechnungen oder große Mengen von Anfragen auf mehrere parallel arbeitende Systeme. Dies kann sehr unterschiedliche Ausprägungen haben. Eine einfache Lastverteilung findet zum Beispiel auf Rechnern mit mehreren Prozessoren statt. Jeder Prozess kann auf einem eigenen Prozessor ausgeführt werden. Man unterscheidet eine Reihe von Algorithmen, genannt Load Balancing Methoden, um diese Verteilung durchzuführen.

\subsection{Ziele}

\subsection{Voraussetzungen}
\begin{itemize}
	\item Grundlagen zu Load Balancing
	\item Java Programmierkenntnisse
\end{itemize}


\subsection{Aufgabenstellung}
Es soll ein Load Balancer mit mindestens 2 unterschiedlichen Load-Balancing Methoden implementiert werden. Eine Kombination von mehreren Methoden ist möglich. Die Berechnung bzw. das Service ist frei wählbar!

Folgende Load Balancing Methoden stehen zur Auswahl:

\begin{itemize}
	\item Weighted Round-Round
	\item Least Connection
	\item Weighted Least Connection
	\item Agent Based Adaptive Balancing / Server Probes
\end{itemize}

Es sollen die einzelnen Server-Instanzen in folgenden Punkten belastet werden können:
\begin{itemize}
	\item Memory (RAM)
	\item CPU Cycles
\end{itemize}

Bedenken Sie dabei, dass die einzelnen Load Balancing Methoden unterschiedlich auf diese Auslastung reagieren werden. Dokumentieren Sie dabei aufkommenden Probleme ausführlich.
\clearpage

