%!TEX root=../document.tex

\section{Einführung}
Diese Übung zeigt die Anwendung von mobilen Diensten.

\subsection{Ziele}
Das Ziel dieser Übung ist eine Webanbindung zur Benutzeranmeldung umzusetzen. Dabei soll sich ein Benutzer registrieren und am System anmelden können.

Die Kommunikation zwischen Client und Service soll mit Hilfe einer REST Schnittstelle umgesetzt werden.

\subsection{Voraussetzungen}
\begin{itemize}
	\item Grundlagen einer höheren Programmiersprache
	\item Verständnis über relationale Datenbanken und dessen Anbindung mittels ODBC oder ORM-Frameworks
	\item Verständnis von Restful Webservices
\end{itemize}


\subsection{Aufgabenstellung}
Es ist ein Webservice zu implementieren, welches eine einfache Benutzerverwaltung implementiert. Dabei soll die Webapplikation mit den Endpunkten /register und /login erreichbar sein.

\textit{Registrierung}\\
Diese soll mit einem Namen, einer eMail-Adresse als BenutzerID und einem Passwort erfolgen. Dabei soll noch auf keine besonderen Sicherheitsmerkmale Wert gelegt werden. Bei einer erfolgreichen Registrierung (alle Elemente entsprechend eingegeben) wird der Benutzer in eine Datebanktabelle abgelegt.

\textit{Login}\\
Der Benutzer soll sich mit seiner ID und seinem Passwort entsprechend authentifizieren können. Bei einem erfolgreichen Login soll eine einfache Willkommensnachricht angezeigt werden.

Die erfolgreiche Implementierung soll mit entsprechenden Testfällen (Acceptance-Tests bez. aller funktionaler Anforderungen mittels Unit-Tests) dokumentiert werden. Verwenden Sie auf jeden Fall ein gängiges Build-Management-Tool (z.B. Maven oder Gradle). Dabei ist zu beachten, dass ein einfaches Deployment möglich ist (auch Datenbank mit z.B. file-based DBMS).
\clearpage
