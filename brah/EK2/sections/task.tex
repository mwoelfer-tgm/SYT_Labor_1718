%!TEX root=../document.tex
\section{Aufgabenstellung}
Sie erhalten ein lauffähiges Neuronales Netzwerk (NN) inklusive Sourcecode.

Adaptieren Sie dieses in Bezug auf Topologie, Trainingsset, usw., sodass es möglich ist, damit die unten stehenden Probleme zu lösen. Trainieren Sie das adaptierten NN entsprechend, zeigen Sie die erfolgreiche Anwendung und dokumentieren Sie diese im Abgabeprotokoll.

Lösen sie ein XNOR mit der Wahrheitstabelle

0 0->1

1 0->0

0 1->0

1 1->1

Lösen Sie mit einem anderen NN ein NAND mit der Wahrheitstabelle

0 0->1

1 0->1

0 1->1

1 1->0

Gehen Sie insbesondere auf die Unterschiede ein, die sich aufgrund der Asymmetrie der Lösungslandschaft für das unterschiedliche Training (Trainingsset, Trainingsreihenfolge, Trainingswiederholungen) der beiden NN’s ergibt und schaffen Sie für NAND eine programmatisch funktionierende Lösung.

Versuchen Sie in einem weiteren Schritt ein NN zu finden mit dem Sie in der Lage sind, die Kombination der beiden Aufgaben zu lösen.

0 0->1 1

1 0->0 1

0 1->0 1

1 1->1 0

Dokumentieren Sie die von Ihnen gefundene Lösung wiederum im Abgabeprotokoll.
\clearpage
