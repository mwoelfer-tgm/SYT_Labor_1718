%!TEX root=../document.tex
\section{Aufgabenstellung}

\subsection{Übungen- Remote Shell}
\subsubsection{SSH-Server Public key Authentifizierung}
Es soll ein SSH-Zugang für eine Authentifizierung mittels Public-Key-Verfahren konfiguriert werden. Dazu soll am Client ein Schlüsselpaar erstellt, der öffentliche Teil der Schlüssel auf den Server übertragen und anschließend der Server für die Schlüssel-Authentifizierung eingerichtet werden. Anschließend soll der Benutzer ohne Login-Passwort am Server anmelden können.

\href{https://www.thomas-krenn.com/de/wiki/OpenSSH\_Public\_Key\_Authentifizierung\_unter\_Ubuntu}{https://www.thomas-krenn.com/de/wiki/OpenSSH\_Public\_Key\_Authentifizierung\_unter\_Ubuntu}

\subsubsection{Konfiguration eines SSH-Tunnels für den Internet-Zugriff über einen remote Server}
Es soll ein sicherer Fernzugriff via TightVNC zum einem Linux-Server ermöglicht werden. Dazu soll SSH-Tunnel eingesetzt werden.

\href{https://www.digitalocean.com/community/tutorials/how-to-install-and-configure-vnc-on-ubuntu-16-04}{https://www.digitalocean.com/community/tutorials/how-to-install-and-configure-vnc-on-ubuntu-16-04}

\href{https://www.theurbanpenguin.com/creating-an-ssh-tunnel-with-putty-to-secure-vnc/}{https://www.theurbanpenguin.com/creating-an-ssh-tunnel-with-putty-to-secure-vnc/}

\subsection{Remote Network}
\subsubsection{Installation und Konfiguration eines VPN Gateway (StrongSwan)}

\begin{itemize}
	\item Konfiguration IPsec Site-to-Site VPN  
	\item Konfiguration  IPsec End-to-Site VPN („Roadwarrior“)
\end{itemize}
\subsubsection{Installation und Konfiguration einer Certificate Authority}
\begin{itemize}
	\item Aktivierung der Authentisierung über Zertifikate 
\end{itemize}

\href{https://console.kim.sg/strongswan-ipsec-vpn-with-pre-shared-key-and-certificates/}{https://console.kim.sg/strongswan-ipsec-vpn-with-pre-shared-key-and-certificates/}

\clearpage