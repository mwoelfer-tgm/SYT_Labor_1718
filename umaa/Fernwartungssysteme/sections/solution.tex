%!TEX root=../document.tex
\section{Ergebnisse}
\subsection{Public Key Authentifizierung}

Client: \verb|10.0.2.7|

Server: \verb|10.0.2.6|
\subsubsection{ssh-key generieren}
Das Schlüsselpaar wird generiert mit \verb|ssh-keygen -b 4096|.
Es wird anschließend nach einem Verzeichnis gefragt, wo der \textbf{private} und \textbf{öffentliche} Schlüssel gespeichert wird.

Der private Schlüssel hat die Bezeichnung \verb|key_rsa| und der öffentliche \verb|key_rsa.pub|.

\begin{minipage}{\linewidth}
	\centering
	\includegraphics[width=0.8\linewidth]{images/generate_key}
	\figcaption{ssh-key wird generiert}
\end{minipage}


\subsubsection{Key auf Server übertragen}
Um den Key auf den Server zu übertragen, müssen Client und Server zuerst in einem Netzwerk sein. Dies wurde umgesetzt mit einem NAT-Netzwerk.

Der öffentliche Schlüssel wurde auf den Server kopiert mit:\\\verb|ssh-copy-id -i .ssh/key_rsa.pub bitte@10.0.2.6|

Es musste lediglich das Passwort des Benutzers am Server angegeben werden.

\begin{minipage}{\linewidth}
	\centering
	\includegraphics[width=0.8\linewidth]{images/transfer_key}
	\figcaption{Key wird auf Remote-Server übertragen}
\end{minipage}


\subsubsection{Auf server zugreifen}
Es kann sich nun am Server eingeloggt werden mit:\\\verb|ssh bitte@10.0.2.6|

Wichtig: Es wird nur mehr nach der Passphrase gefragt und nicht nach dem öffentlichen Schlüssel.

\begin{minipage}{\linewidth}
	\centering
	\includegraphics[width=0.8\linewidth]{images/access_server}
	\figcaption{Es wird auf den Server zugegriffen}
\end{minipage}

Auch zu beachten ist, nach dem ersten verbinden durch ssh wird auch nicht mehr nach der Passphrase gefragt.

\clearpage
\subsection{SSH-Tunnel Zugriff über Remote-Server}
\subsubsection{Server vorbereiten}
Um auf den Server über eine grafische Oberfläche zuzugreifen muss zuerst ein Desktop Environment und ein VNC-Server installiert werden.

\verb|sudo apt-get install xfce4 xfce4-goodies tightvncserver|

Nach der Installation wird der VNC-Server konfiguriert mit dem Befehl \verb|vncserver|:

\begin{minipage}{\linewidth}
	\centering
	\includegraphics[width=0.8\linewidth]{images/vncserver}
	\figcaption{Der VNC-Server wird konfiguriert}
\end{minipage}

Tatsächlich wird nur nach einem Passwort für Remote Control (Tastatur und Maus) und View-Only definiert.

Nun soll noch geändert werden was passiert wenn ein VNC-Server gestartet wird. Dafür wird zuerst der service gekilled mit \verb|vncserver -kill :1|:

\begin{minipage}{\linewidth}
	\centering
	\includegraphics[width=0.8\linewidth]{images/kill_service}
	\figcaption{Der VNC-Server Service wird gekilled für Bearbeitung}
\end{minipage}

Anschließend wird das File geöffnet welches bestimmt welche Aktionen durchgeführt werden wenn der VNC-Server gestartet wird mit \verb|sudo nano ~/.vnc/xstartup|

Falls Inhalt bereits vorhanden ist, wird dieser aus dem File gelöscht, und folgender Inhalt wird hinzugefügt:

\begin{lstlisting}[language=bash]
#!/bin/bash
xrdb $HOME/.Xresources
startxfce4 &
\end{lstlisting}

Der Befehl \verb|xrdb $HOME/.Xresources| kümmert sich darum, dass der User bestimmte Einstellungen zu der grafischen Oberfläche machen kann wie Farben ändern, Designs oder Schriftarten.

Der zweite Befehl \verb|startxfce4 &| startet lediglich Xfce, was die komfortable Bearbeitung durch die grafische Oberfläche ermöglicht.


Es müssen diesem File auch noch bestimmte Privilegien zugewiesen werden: \\\verb|sudo chmod +x ~/.vnc/xstartup|

Der service kann nun durch \verb|vncserver| gestartet werden, der output sollte folgendermaßen aussehen:

\begin{minipage}{\linewidth}
	\centering
	\includegraphics[width=0.8\linewidth]{images/start_service}
	\figcaption{Der VNC-Server wird wieder gestartet}
\end{minipage}

\subsubsection{Client vorbereiten}
In diesem Fall stellt die lokale Windows-Maschine den Client dar. Es werden 2 Programme benötigt, \verb|PuTTY| und \verb|TightVNC Viewer|.

Mit \verb|PuTTY| wird der SSH-Tunnel geöffnet und \verb|TightVNC Viewer| greift über diesen Tunnel auf die grafische Oberfläche des Servers zu.

\subsubsection{Mit Client zu Server verbinden}
Zuerst wird mit \verb|PuTTY| der Tunnel erstellt.

Dazu wird das Programm geöffnet und als IP-Adresse wird \verb|bitte@10.0.106.174| angegeben und es soll über SSH verbunden werden:

\begin{minipage}{\linewidth}
	\centering
	\includegraphics[width=0.8\linewidth]{images/user_ip}
	\figcaption{Es werden user, Ip-Adresse und Übertragungsart angegeben}
\end{minipage}

Der nächste Schritt ist es nun den Tunnel zu definieren. Diese Einstellung ist zu finden unter \verb|Connection|$\rightarrow$\verb|SSH|$\rightarrow$\verb|Tunnels|.

Hier wird als \verb|Source Port| ein beliebiger Port angegeben (Für das Beispiel wurde 9090 gewählt) und als \verb|Destination| wird der VNC-Server angegeben, mit der Adresse \verb|localhost:5901|. Es kann auf \verb|Add| gedrückt werden und man sollte folgendes Ergebnis sehen:

\begin{minipage}{\linewidth}
	\centering
	\includegraphics[width=0.8\linewidth]{images/ports}
	\figcaption{Es wurde ein Tunnel zum VNC Server hinzugefügt}
\end{minipage}

Nun kann auf \verb|Open| gedrückt werden und man sollte nach dem User-Passwort gefragt werden. Nach der Passworteingabe sollte man die Konsole des Servers sehen:

\begin{minipage}{\linewidth}
	\centering
	\includegraphics[width=0.8\linewidth]{images/console}
	\figcaption{Es wurde zum Server verbunden}
\end{minipage}

Das wichtige ist nun im \verb|Event Log| zu sehen, wenn man diesen öffnet (Rechtsklick auf Statusleiste$\rightarrow$Event Log) sollte man folgende 3 Zeilen sehen:

\begin{minipage}{\linewidth}
	\centering
	\includegraphics[width=0.8\linewidth]{images/event_log}
	\figcaption{Ein Channel mit Port forwarding zum Server wurde geöffnet}
\end{minipage}

Nun kann man den \verb|TightVNC Viewer| öffnen. Im Feld \verb|Remote Host| gibt man nun \verb|localhost:9090| an und wird wieder nach dem Passwort gefragt. Nach Eingabe des Passworts sollte die grafische Oberfläche des Server zu sehen sein:

\begin{minipage}{\linewidth}
	\centering
	\includegraphics[width=0.8\linewidth]{images/gui}
	\figcaption{Es kann über eine grafische Oberfläche auf den Server zugegriffen werden}
\end{minipage}

Es wird nun mit dem Viewer auf den verschlüsselten SSH-Tunnel zugegriffen welcher eine Verbindung zum Server hat!

\subsection{ipsec}
\subsubsection{Passwort}
Zuerst wurde auf den beiden virtuellen Maschinen \verb|strongswan|

\begin{lstlisting}[language=Java]
sudo apt-get install strongswan
\end{lstlisting}

Zusätzlich wird auch \verb|haveged| installiert und der Dienst gestartet:

\begin{lstlisting}[language=Java]
sudo apt-get install haveged
sudo systemctl enable haveged
sudo systemctl start haveged
\end{lstlisting}

Der nächste Schritt ist es auf den beiden Maschinen den gleichen \textbf{PSK} (\textbf{P}re \textbf{S}hared \textbf{K}ey) zu definieren im \verb|/etc/ipsec.secrets| file:


\begin{tabular}{p{8cm} p{8cm}}
	left  &   right\\
	\begin{lstlisting}
	10.0.2.8 10.0.2.9 : PSK 'schueler'
	\end{lstlisting}&
	\begin{lstlisting}
	10.0.2.9 10.0.2.8 : PSK 'schueler'
	\end{lstlisting}
\end{tabular} 




Zum Schluss müssen noch die individuellen Config Files zu bearbeiten, diese liegen jeweils in \verb|/etc/ipsec.conf|:

\begin{tabular}{p{8cm} p{8cm}}
	left  &   right\\
	\begin{lstlisting}
conn A_TO_B
	authby=secret
	left=10.0.2.8
	leftsubnet=10.0.2.0/24
	right=10.0.2.9
	rightsubnet=10.0.2.0/24
	ike=aes256-sha2_256-modp1024!
	esp=aes256-sha2_256!
	keyingtries=0
	ikelifetime=1h
	lifetime=8h
	dpddelay=30
	dpdtimeout=120
	dpdaction=restart
	auto=start
	\end{lstlisting}&
	\begin{lstlisting}
conn B_TO_A
	authby=secret
	left=10.0.2.9
	leftsubnet=10.0.2.0/24
	right=10.0.2.8
	rightsubnet=10.0.2.0/24
	ike=aes256-sha2_256-modp1024!
	esp=aes256-sha2_256!
	keyingtries=0
	ikelifetime=1h
	lifetime=8h
	dpddelay=30
	dpdtimeout=120
	dpdaction=restart
	auto=start
	\end{lstlisting}
\end{tabular} 

Danach muss ipsec auf beiden Maschinen neu gestartet werden mit \verb|sudo ipsec restart| und danach kann schon der Tunnel getestet werden mit \verb|sudo ipsec up A_TO_B| auf Maschine A bzw. \verb|sudo ipsec up B_TO_A| auf Maschine B. Tatsächlich wird aber der Befehl nur auf einer Maschine ausgeführt da der Tunnel zwischen den beiden Geräten aufgebaut wird.

\begin{minipage}{\linewidth}
	\centering
	\includegraphics[width=0.8\linewidth]{images/left}
	\figcaption{Es wurde erfolgreich der Tunnel zwischen den beiden Geräten aufgebaut}
\end{minipage}


Um zu testen ob die Pakete mit \textbf{ESP} verschlüsselt werden, wird von einer Maschine auf die andere gepingt, während die andere \verb|sudo tcpdump esp| ausführt.

Es sollten dauerhaft \textbf{responses} und \textbf{reqeuests} zu sehen sein, welche mit \textbf{ESP} verschlüsselt wurden. 

\begin{minipage}{\linewidth}
	\centering
	\includegraphics[width=0.8\linewidth]{images/tcpdump}
	\figcaption{Pakete werden mit ESP verschlüsselt}
\end{minipage}

\subsubsection{Zertifikate}
Die Authentifizierung um einen IPsec Tunnel zu erstellen, kann auch über Zertifikate (erinnert an LDAP) umgesetzt werden.

Der erste Schritt ist es, alle öffentlichen Zertifikate zu erstellen, sowohl auf \textbf{HQ}, als auch \textbf{REMOTE}:

\begin{lstlisting}[language=bash]
cd /etc/ipsec.d

#Create private key:
ipsec pki --gen --type rsa --size 4096 --outform pem > private/strongswanKey.pem
chmod 600 private/strongswanKey.pem

#Generate a self signed root CA certificate using above private key:
ipsec pki --self --ca --lifetime 3650 --in private/strongswanKey.pem --type rsa --dn "C=Kim, O=Kim, CN=Kim Root CA" --outform pem > cacerts/strongswanCert.pem

# View the X.509 certificate properties
ipsec pki --print --in cacerts/strongswanCert.pem

#Generate private key for this VPN host server
ipsec pki --gen --type rsa --size 4096 --outform pem > private/vpnHostKey.pem
chmod 600 private/vpnHostKey.pem

#Generate this VPN host server cert using earlier CA
ipsec pki --pub --in private/vpnHostKey.pem --type rsa | ipsec pki --issue --lifetime 730 --cacert cacerts/strongswanCert.pem --cakey private/strongswanKey.pem --dn "C=Kim, O=Kim, CN=vpn.example.com.sg" --san vpn.example.com.sg --san vpn2.example.com.sg --san xx.xxx.xxx.xxx --san @xx.xxx.xxx.xxx --flag serverAuth --flag ikeIntermediate --outform pem > certs/vpnHostCert.pem

#View newly generated certificate
ipsec pki --print --in certs/vpnHostCert.pem
\end{lstlisting}

Nun wird nur auf \textbf{REMOTE}, also Client, die privaten Zertifikate erstellt:

\begin{lstlisting}[language=bash]
#Genrate Private key for client
cd /etc/ipsec.d
ipsec pki --gen --type rsa --size 2048 --outform pem > private/KimKey.pem
chmod 600 private/KimKey.pem

#Generate Cert for client, signed by our root ca
ipsec pki --pub --in private/KimKey.pem --type rsa | ipsec pki --issue --lifetime 730 --cacert cacerts/strongswanCert.pem --cakey private/strongswanKey.pem --dn "C=Kim, O=Kim, CN=kim@example.com" --san "kim@example.org" --san "kim@example.net" --san "kim@xxx.xx.xx.xx" --outform pem > certs/KimCert.pem
\end{lstlisting}

Nun das auf beiden Maschinen die Zertifikate erstellt wurden, kann ipsec konfiguriert werden.

Zuerst wird in \verb|/etc/ipsec.secrets| das jeweilige Zertifikat für die Authentifizierung angegeben:

\begin{tabular}{p{8cm} p{8cm}}
	HQ  &   REMOTE\\
	\begin{lstlisting}
	: RSA vpnHostKey.pem
	\end{lstlisting}&
	\begin{lstlisting}
	: RSA KimKey.pem
	\end{lstlisting}
\end{tabular} 

Als nächstes werden die beiden Konfiguration erstellt und deklariert in \verb|/etc/ipsec.conf|:

\begin{tabular}{p{8cm} p{8cm}}
	HQ  &   REMOTE\\
	\begin{lstlisting}
conn HQ_TO_REMOTE
	keyexchange=ikev2
	leftcert=vpnHostCert.pem
	left=10.0.2.15
	# leftid=%any Can skip. Pulls from its cert
	leftsubnet=10.0.2.0/24
	right=10.0.2.6
	rightid=%any #"C=Kim, O=Kim, CN=userX@gmail.com"
	rightsubnet=10.0.2.0/24
	auto=add
	\end{lstlisting}&
	\begin{lstlisting}
conn REMOTE_TO_HQ
	keyexchange=ikev2
	leftcert=KimCert.pem
	left=10.0.2.6
	leftid="C=Kim, O=Kim, CN=userX@gmail.com"
	leftsubnet=10.0.2.0/24
	right=10.0.2.15
	rightid=%any
	rightsubnet=10.0.2.0/24
	auto=add
	\end{lstlisting}
\end{tabular} 
\clearpage

Bei Ausführung des Befehls \verb|sudo ipsec HQ_TO_REMOTE up|, kann man folgendes sehen:

\begin{minipage}{\linewidth}
	\centering
	\includegraphics[width=0.8\linewidth]{images/ipsec_cert1}
	\figcaption{Tunnel wurde erfolgreich erstellt mit Zertifikaten}
\end{minipage}

Und beim pingen der Maschine, kann man mit \verb|sudo tcpdump esp| wieder sehen, dass die Pakete verschlüsselt werden:

\begin{minipage}{\linewidth}
	\centering
	\includegraphics[width=0.8\linewidth]{images/tcpdump2}
	\figcaption{Pakete werden verschlüsselt mit Authentifizierung über Zertifikate}
\end{minipage}