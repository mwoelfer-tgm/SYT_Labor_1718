%!TEX root=../document.tex

\section{Aufgabenstellung}
Die Aufgabe ist im Detail am Ende des Dokumentes "InterruptAdvanced.pdf" beschrieben. In diesem Dokument ist davor auch die technik der Interrupt-Priorisierung einigermaßen detailloiert beschrieben. 

Führe die Aufgaben wie im Dokument beschrieben durch und schreibe ein zugehöriges Protokoll. GIb den Sourceode und das Protokoll ab.
\subsection{Teil (a): Interrupts für Blinken und Toggeln}
Erstelle ein einfaches Programm, wobei zwei LEDs blinken. Verwende hierbei für eine LED
die Funktion HAL\_Delay im Hauptprogramm. Diese nutzt wiederum implizit den SystickInterrupt.
Die zweite LED wird direkt über den Systick-Callback gesteuert. Eine weitere LED
wird mit dem Taster getoggelt. Alles zusammen funktioniert einwandfrei. Beide InterruptRoutinen
(Systick und EXTI0) werden schnell wieder verlassen, sodass es dabei zu keinen
Problemen kommt.
\subsection{Teil (b): Blinken innnerhalb einer Interruptroutine}
Wenn der Taster des Boards gedrückt wird, so möge eine LED drei Mal blinken. Die
Steuerung für das Blinken möge in der Interruptroutine des Tasters selbst erfolgen. Das
Blinken soll mit HAL\_Delay gesteuert werden. Erst nach dem Blinken soll die InterruptRoutine
verlassen werden.
Die Priorität und des Taster-Interrupts muss man ordentlich konfigurieren. In HAL\_Init wird
die Priorität des Systick-Interrupts auf niedrig (15) gesetzt. Man muss die Priorität von
Systick auf hoch setzen (z.B.: 0) und die Priorität des Taster - Interrupts auf einen etwas
niedrigeren Wert (z.B.: 1). Ändert man die Priorität nicht, so bleibt das Programm ewig in der
Interrupt-Routine des Buttons hängen!
\subsection{Teil (c): Eigenen Interrupt definieren und via Software auslösen}
Trage deine eigene Interrupt-Routine (Name könnte sein <deinName>\_IRQHandler) in einer
feien Stelle der Interrupttabelle ein. Definiere lokal die zugehörige Interrupt-Nummer
(<deinName\_IRQn). Die Interrupt-Routine möge einfach nur ein LED toggeln. Aktiviere dein
Interrupt in der Interrupttabelle. Löse nun den Interrupt in der Callback-Routine für den
Taster aus. Wenn die Priorität des selbstdefinierten Interrupts höher ist als die des Tasters,
so wird er sofort ausgeführt. Wenn die Priorität gleich oder niedriger ist, so wird der
selbstdefinierte Interrupt erst ausgeführt, nachdem der Taster-Interrupt beendet ist.
\subsection{Teil (d): Externen Schalter entprellen}
Unter Tasterprellen versteht man folgendes: Ein Taster kann beim Drücken und Loslassen
kurze Zeit vibrieren und dabei öfters die Tasterpins verbinden und wieder trennen. Bei einer
Interruptsteuerung, welche auf rising und/oder falling edges reagiert, kann dies zu
oftmaligen Interrupts und damit zu undefinierten Software-Zuständen führen. Ein Taster ist
entprellt, wenn solche Vibrationen keine Auswirkungen auf die Funktionalität des
Gesamtsystems haben.
Wir entprellen einen externen Taster folgendermaßen: Wir warten im Interrupt wenige
Millisekunden und prüfen dann den Zustand des zugehörigen Pins. Ist der Zustand auf 1,
dann haben wir einen rising-edge Interrupt, ist der Zustand auf 0 so handelt es sich um
einen falling-edge Interrupt. Wir ignorieren letzteren!
Mit einem solchen entprellten externen Schalter toggeln wir eine LED.
Wiederum muss die Priorität des Systick-Interrupts höher sein als die Priorität des Interrupts
des externen Tasters.
\subsection{Teil (e): 2 Schalter mit unterschiedlichen Prioritäten}
Realisiere folgendes:
\begin{itemize}
	\item Ein Blinklicht in der main-Funktion mittels HAL\_Delay.
	\item den Systick-Interrupt mit hoher Priorität
	\item Bei Betätigen des Taster am Board soll ein zweites LED 5 Mal blinken, realisiert
	mittels HAL\_Delay.
	\item Einen externen (entprellten) Schalter, welcher ein weiteres LED 5 Mal blinken lässt,
	wiederum mit HAL\_Delay realisiert. Die Priorität ist hier niedriger als beim Taster am
	Board.
\end{itemize}

Wenn sich die Preemptive-Priorität sieht man hier folgendes:
\begin{itemize}
	\item Das Betätigen eines Tasters unterbricht den Blinker der main-Funktion.
	\item Das Betätigen des Board-Tasters unterbricht das Blinken des externen Schalters, aber
	nicht umgekehrt. Es wird hier angenommen, dass sich die Preemptive-Priorität der
	beiden Schalter unterscheidet.
\end{itemize} 
Führe nun dieselbe Aufgabe noch einmal durch. Nun möge die preemptive-Priorität der
beiden Schalter gleich sein, sie sollen aber unterschiedliche Sub-Priorität haben. In diesem
Fall wird ein laufender Interrupt nicht unterbrochen, der andere Interrupt kommt erst nach
Beendigung des ersten Interrupt s an die Reihe. Die unterschiedliche Sub-Priorität kann man
folgendermaßen feststellen:
\begin{itemize}
	\item Betätige einen Schalter, sodass der entsprechende Blinker läuft.
	\item Betätige nun beide Taster je einmal.
	\item Nachdem die eine Blinksequenz fertig ist, wird der Interrupt mit der höheren Subpriorität
	ausgewählt, unabhängig von der Reihenfolge, in welcher die beiden Schalter betätigt
	wurden.
\end{itemize}
\subsection{Teil (f): Drei Blinklichter mit Systick-Interrupt-Steuerung}
Realisiere die obige Aufgabe (drei Blinklichter, ein ewiges und zwei auf Tastendruck) nun
noch einmal, dieses Mal jedoch ohne HAL\_Delay. Alle Blinker sollen mit dem SystickInterrupt
gesteuert werden. In diesem Fall ist die Priorität der Interrupts egal, dabei jedem
Interrupt der Mikrocontroller nur ganz kurze Zeit im Interrupt-Modus bleibt. Die mainFunktion
enthält am Ende eine leere Endlosschleife, davor müssen die Interrupts und die
LEDs konfiguriert werden.
\clearpage
